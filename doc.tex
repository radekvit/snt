\documentclass[12pt, a4paper]{article}
\usepackage[czech]{babel}
\usepackage[utf8]{inputenc}
\usepackage{amsmath}
\usepackage{amsthm}
\usepackage{amssymb}
\usepackage{enumitem}
\usepackage{algorithm}
\usepackage{algpseudocode}
\usepackage{mathtools}
\usepackage{listings}
\usepackage{color}
\PassOptionsToPackage{hyphens}{url}\usepackage{hyperref}

\usepackage{calrsfs}

\definecolor{light-gray}{gray}{0.85}
\lstset{
    numbers=left,
    breaklines=true,
    backgroundcolor=\color{light-gray},
    tabsize=2,
    basicstyle=\ttfamily,
}

\DeclareMathAlphabet{\pazocal}{OMS}{zplm}{m}{n}

\newcommand{\Lb}{\pazocal{L}}

%define Input and Output commands in algorithmicx
\algnewcommand\algorithmicinput{\textbf{Vstup:}}
\algnewcommand\Input{\item[\algorithmicinput]}

\algnewcommand\algorithmicoutput{\textbf{Výstup:}}
\algnewcommand\Output{\item[\algorithmicoutput]}

\let\oldemptyset\emptyset
\let\emptyset\varnothing
\let\oldepsilon\epsilon
\let\epsilon\varepsilon
\let\emptystring\varepsilon

\def\CC{{C\nolinebreak[4]\hspace{-.05em}\raisebox{.4ex}{\tiny\bf ++}}}
\def\CS{{\settoheight{\dimen0}{C}C\kern-.05em \resizebox{!}{\dimen0}{\raisebox{\depth}{\#}}}}

\author{Radek Vít\\ \texttt{xvitra00}}
\title{
	\includegraphics[scale=0.6]{FIT_barevne_PANTONE_CZ.pdf}\\
	SNT\\
	University Course Timetabling Problem:\\ Optimalizace pomocí algoritmu páření včel
}

\begin{document}
\newpage
\maketitle

\section{Úvod}
% proc tato prace
% muj prinos
% v jakem prostredi
\section{Popis problému}
Problém rozvrhnutí univerzitních přednášek je definovaný jako %TODO citace z HoneyBee
přiřazení daného počtu přednášek (také courses) do daného počtu vyučovacích hodin (také timeslots) a místností,
kde je splněná množina tvrdých omezení. %citovat Socha
V této práci se zaměřuji na tento konkrétní model úlohy, kterou tvoří:
\begin{itemize}
  \item množina přednášek $C$
  \item množina vyučovacích hodin $T = \{t_n | n \in \{0, \dots, 45)\}\}$ (9 pro každý den)
  \item množina místností $R = \{r_j | (j = 0, \dots, R)\}$
  \item množina vlastností místností $F$
  \item množina studentů $S$
\end{itemize}
Každý student $s \in S$ chodí na přednášky $C_s \subseteq C$. Každá místnost $r \in R$ má některé z vlastností $f_r \subseteq F$ a nějakou kapacitu $s_r \in \mathbb{N}$.

Úlohou je přiřadit každou přednášku $c_i$ do hodiny $t_n$ a místnosti $r_j$ tak, aby byla splněna následující tvrdá omezení:
\begin{itemize}
  \item Každý student může mít v každé hodině nejvýše jednu přednášku.
  \item Přiřazená místnost musí splňovat vlastnosti vyžadované přednáškou.
  \item Přiřazená místnost musí mít dostatečnou kapacitu na všechny studenty účastnící se přednášky.
  \item V každé místnosti v jedné vyučovací hodině se může konat nejvýše jedna přednáška.
\end{itemize}
Kvalita řešení se poté hodnotí podle počtu porušení měkkých omezení:
\begin{itemize}
  \item Student by neměl mít přednášku v poslední hodině dne.
  \item Každý student by měl mít nejvýše 2 navazující přednášky.
  \item Student by měl mít za den více než jednu přednášku.
\end{itemize}
% ozdrojovana fakta
% pojmy ze SNT CITOVAT!!!
\section{Algoritmus páření včel pro problém rozvrhnutí univerzitních přednášek}
% zdrojuj
% TODO cituj honey-bee
Algoritmus páření včel je jedním z optimalizačních algoritmů inspirovaných přírodou.
Včelí kolonie sestává z královny (nejlepší nalezené řešení), trubců (možná řešení), pracovníků (heuristická vylepšení)
a mladušek (potenciální řešení). Popis algoritmu i jeho modifikace je převzatá z článku "A honey-bee mating optimization algorithm for educational timetabling problems" (viz. \cite{HoneyBee}).
Základní algoritmus 

Na začátku je vytvořena populace trubců a nejlepší z nich je zvolený za královnu.
Poté se po daný počet opakování opakuje cyklus páření a vytváření nové generace.

Královně je přiřazena energie $e_{t0} = random \in <0.5, 1>$.
Poté královna náhodně navštivuje trubce, dokud se nespářila s předem daným počtem, nebo dokud jí nedojde energie ($e_t \leq \epsilon$).
Při každé návštěvě trubce je pravděpodobnost spáření se s daným trubcem závislá na jejich fitness funkci $f$:
$p(queen, drone) = e^{\frac{f(queen) - f(drone)}{energy(t)}}$.
Na začátku letu má královna vysokou energii, a pravděpodobnost páření je tak vyšší.
Po každém pokusu o páření je zmenšena energie královny ($e_{t+1} = e_{t} * \Alpha$, $\Alpha = 0.9$.

Po dokončení páření tvoří královna mladušky křížením své genetické informace s genetickou informací trubců, se kterými se královna spářila (jedna mladuška pro každého trubce).
Mladušce je přiřazena včela pracovník (heuristicka) pro zlepšení fitness funkce mladušky.
Pokud je mladuška lepší než královna, vymění si s ní místo a stává se tedy novou královnou.
Proti algoritmu z článku je implementovaný přístup pozměněný: po nalezení lepšího řešení jsou nové mladušky vytvořeny křížením
nové královny se zbývajícími geny trubců.

V původní verzi algoritmu se pro další cyklus páření a vytváření mladušek trubci nemění, nicméně toto vede k příliš rychlé konvergenci
a původní populace trubců je po pozměnění pomocí Kempe-chain struktur nahrazena populací mladušek.
Tento přístup zabraňuje předčasné konvergenci algoritmu.

\subsection{Vytvoření počáteční populace}
Na začátku je potřeba získat několik řešení, která neporušují žádná tvrdá omezení.
Tato řešení jsou poté použita jako první generace trubců a nejlepší z nich jako královna.

Pro vytvoření počáteční generace využíváme tří heuristik pro barvení grafu:
\begin{itemize}
  \item SD: přednášky jsou seřazeny vzestupně podle počtu vyučovacích hodin, do kterých je lze přiřadit
  \item LD: přednášky jsou seřazeny sestupně podle počtu všech přednášek, se kterými mají společné studenty
  \item LE: přednášky jsou seřazeny sestupně podle počtu studentů, kteří se této přednášky účastní
\end{itemize}
Všechny události nejdříve seřadíme podle těchto tří heuristik (s prioritami $SD > LD > LE$). Náhodně vybereme některou z platných vyučovacích hodin pro první přednášku v seznamu a zařadíme
ji do platné místnosti s nejmenší velikostí a následně nejmenším počtem vlastností.
Události znovu seřadíme a opakujeme náhodný výběr vyučovacích hodin a přiřazení do místnosti.
Pokud jsme nepřiřadili všechny události, a první událost v seznamu nemá žádnou přiřaditelnou vyučovací hodinu nebo ji nelze přiřadit do místnosti ve vybrané vyučovací hodině, dosavadní částečné řešení
zahodíme a začneme znovu.
Experimentálně se ukázalo, že není potřeba pro případ selhání zavádět prohledávání stavového prostoru (např. backtracking):
při generování počátečních řešení se díky heuristikám jen vzácně vyskytnul případ, kdy nebylo možné zbývajícím hodinám přiřadit místo,
a nedochází tedy několikanásobné zahazování částečných řešení.

Tento přístup negarantuje, že existuje nějaké splnitelné řešení; v případě, že žádné neexistuje, zůstane algoritmus v nekonečné smyčce.
V žádné z testovaných úloh tento problém nenastal, a řešení nesplnitelných úloh není předmětem tohoto algoritmu.

\subsubsection{Křížení královny s trubcem}
Jako gen řešení je braná jedna vyučovací hodina, tj. všechny přednášky přiřazené do místností v jedné hodině v jednom dni.
Pro křížení královny s trubcem je použitý následující postup:
Z královny a trubce je vybráno 8 genů a výsledek se vytvoří jako kopie královny.
Pro každou dvojici genů $(q_k, t_k)$ se provede přesouvání přednášek z genu trubce do nového genu královny.
Po jednom se vybírají přiřazené přednášky z genu trubce a zařazují se do genu královny.
Pokud se již přednáška v genu královny vyskytuje, nebo by přemístěním této přednášky vznikl konflikt, nebo pokud nelze nalézt místnost, do které by šla tato přednáška zařadit,
předmět se nepřesouvá. Pokud nevznikne konflikt a lze nalézt místnost, je tato přednáška odstraněna ze svého původního umístění z genu královny.

\subsubsection{Zlepšení mladušky}
Po dokončení křížení je mladuška vylepšena pracovníkem.
Pro vylepšení podle použitého algoritmu z \cite{HoneyBee} je použité lokální prohledávání stavového prostoru.
Opakovaně se vybere náhodná přednáška a přesune se do náhodné vyučovací hodiny, která nenaruší tvrdá omezení.
Pokud tato změna zmenší počet porušení měkkých omezení, je přijata, pokud ne, je řešení zachováno ve své původní podobě.

\subsubsection{Kempe-chain stuktury}
Před použitím mladušek jako nových trubců je zvýšený jejich počet a jsou pozměněné pomocí Kempe-chain struktur.
Mladušky se duplikují, aby jejich celkový počet odpovídal požadované velikosti populace trubců a následně jsou promíchány.

Kempe-chain struktura (viz. \cite{Kempe})
pochází z barvení grafů: jde o největší souvislou komponentu sousedících uzlů, které jsou obarveny střídavě barvou $a$ a $b$. Poté je možné v této struktuře prohodit barvy $a$ a $b$.
Pro problém plánování přednášek lze najít Kempe-chain strukturu následovně:
Vybereme dvě vyučovací hodiny $x$ a $y$. Náhodně vybereme některou z přednášek přiřazenou do $x$ a vložíme ji do množiny $K_x$. Poté nalezneme všechny přednášky z $y$, které jsou v konfliktu s přednáškami z množiny $K_x$ a přidáme je do množiny $K_y$.
Poté střídavě přidáváme přednášky z $x$ do $K_x$, které mají konflikt s některou přednáškou z $K_y$, a přidáváme přednášky z $y$ do $K_y$, které mají konflikt s některou přednáškou z $K_x$.
Cyklus přerušíme, když se ustálí velikost $K_x$ i $K_y$. Poté můžeme z $x$ a $y$ uvolnit místnosti zabrané přednáškami z $K_x$ a $K_y$ a přiřadit do $x$ přednášky z $K_y$ a do $y$ přednášky z $K_x$.
Pokud nelze pro některou z přednášek nalézt volnou místnost, musí být celá změna zamítnuta a musí být případně zvolené jiné dvě vyučovací hodiny.

\section{Implementace}
Optimalizační nástroj byl implementovaný v \CC{}17.
Třídy modelující problém jsou definované a implementované v souboru \texttt{course_timetabling.h}.
Samotný algoritmus, jak je popsaný v tomto článku, je implementovaný ve třídě \texttt{HbmoEtp} a jeho spuštění
implementuje metoda \texttt{run}.

Použité parametry algoritmu se nacházejí v souboru \texttt{config.h}. Úpravou tohoto souboru a novou kompilací
lze pozměnit parametry samotného algoritmu: počet iterací páření, koeficient úbytku energie, počet trubců, počet mladušek (pro tuto implementaci by měl odpovídat čtvrtině počtu trubců),
počet křížených genů a počet iterací heuristického zlepšení mladušek. Nastavit lze také hranici kvality královny, po jejímž dosažení se algoritmus předčasně ukončí.

Program čte problém se standardního vstupu ve formátu specifikovaném na adrese \url{http://sferics.idsia.ch/Files/ttcomp2002/IC_Problem/node7.html}
a výsledek vypisuje na standardní výstup.
V průběhu provádění algoritmu jsou na chybový výstup vypisované doprovodné informace o nejlepším dosaženém výsledku.

\section{Experimenty}
S výsledným optimalizačním nástrojem
\section{Závěr}
%shrnout vysledky a experimenty
\end{document}