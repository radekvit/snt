\documentclass[12pt, a4paper]{article}
\usepackage[czech]{babel}
\usepackage[utf8]{inputenc}
\usepackage{amsmath}
\usepackage{amsthm}
\usepackage{amssymb}
\usepackage{enumitem}
\usepackage{algorithm}
\usepackage{algpseudocode}
\usepackage{mathtools}
\usepackage{listings}
\usepackage{color}
\usepackage{pgfplots}
\PassOptionsToPackage{hyphens}{url}\usepackage{hyperref}

\usepackage{calrsfs}

\definecolor{light-gray}{gray}{0.85}
\lstset{
    numbers=left,
    breaklines=true,
    backgroundcolor=\color{light-gray},
    tabsize=2,
    basicstyle=\ttfamily,
}

\DeclareMathAlphabet{\pazocal}{OMS}{zplm}{m}{n}

\newcommand{\Lb}{\pazocal{L}}

%define Input and Output commands in algorithmicx
\algnewcommand\algorithmicinput{\textbf{Vstup:}}
\algnewcommand\Input{\item[\algorithmicinput]}

\algnewcommand\algorithmicoutput{\textbf{Výstup:}}
\algnewcommand\Output{\item[\algorithmicoutput]}

\let\oldemptyset\emptyset
\let\emptyset\varnothing
\let\oldepsilon\epsilon
\let\epsilon\varepsilon
\let\emptystring\varepsilon

\def\CC{{C\nolinebreak[4]\hspace{-.05em}\raisebox{.4ex}{\tiny\bf ++}}}
\def\CS{{\settoheight{\dimen0}{C}C\kern-.05em \resizebox{!}{\dimen0}{\raisebox{\depth}{\#}}}}

\bibliographystyle{ieeetr}

\author{Radek Vít\\ \texttt{xvitra00}}
\title{
	\includegraphics[scale=0.6]{FIT_barevne_PANTONE_CZ.pdf}\\
	SNT\\
	University Course Timetabling Problem:\\ Optimalizace pomocí algoritmu páření včel
}

\begin{document}
\newpage
\maketitle

\section{Úvod}
% proc tato prace
Problém rozvrhnutí univerzitních přednášek je jedním z běžně řešených optimalizačních problémů.

% muj prinos
Implementoval jsem algoritmus popisovaný v článku "A honey-bee mating optimization algorithm for educational timetabling problems"
(viz. \cite{HoneyBee}) a provedl jsem experimenty, kde jsem porovnával výsledky proti jiným algoritmům implementovanými
kolegy v rámci předmětu SNT. Znázorňuji také na jednom experimentu, jak se mění nejlepší nalezený výsledek v závislosti na počtu provedených iterací hlavní smyčky algoritmu (pářicích běhů).

% v jakem prostredi
Projekt byl implementovaný v \CC{}17 a je implementovaný jen pomocí standardní knihovny, je tedy přeložitelný a spustitelný na jakémkoliv stroji s verzí překladače \CC{}, která podporuje \CC{}17.
\section{Popis problému}
Problém rozvrhnutí univerzitních přednášek (viz. \cite{HoneyBee}) je definovaný jako 
při\-řa\-ze\-ní daného počtu přednášek (také courses) do daného počtu vyučovacích hodin (také timeslots) a místností,
kde je splněná množina tvrdých omezení. %citovat Socha
V této práci se zaměřuji na tento konkrétní model úlohy, kterou tvoří:
\begin{itemize}
  \item množina přednášek $C$
  \item množina vyučovacích hodin $T = \{t_n | n \in \{0, \dots, 45)\}\}$ (9 pro každý den)
  \item množina místností $R$
  \item množina vlastností místností $F$
  \item množina studentů $S$
\end{itemize}
Každý student $s \in S$ chodí na přednášky $c_s \subseteq C$. Každá místnost $r \in R$ má vlastnosti $f_r \subseteq F$ a kapacitu $s_r \in \mathbb{N}$.

Úlohou je přiřadit každou přednášku $c \in C$ do některé hodiny $t_n \in T$ a místnosti $r_j \in R$ tak, aby byla splněna následující tvrdá omezení:
\begin{itemize}
  \item Každý student může mít v každé hodině nejvýše jednu přednášku.
  \item Přiřazená místnost musí splňovat vlastnosti vyžadované přednáškou.
  \item Přiřazená místnost musí mít dostatečnou kapacitu na všechny studenty účastnící se přednášky.
  \item V každé místnosti v jedné vyučovací hodině se může konat nejvýše jedna přednáška.
\end{itemize}
Kvalita řešení se poté hodnotí podle počtu porušení měkkých omezení:
\begin{itemize}
  \item Student by neměl mít přednášku v poslední hodině dne.
  \item Každý student by měl mít nejvýše 2 navazující přednášky.
  \item Pokud má student v jednom dni alespoň jednu přednášku, měl by mít alespoň dvě.
\end{itemize}
\section{Algoritmus páření včel pro problém rozvrhnutí univerzitních přednášek}
Algoritmus páření včel je jedním z optimalizačních algoritmů inspirovaných přírodou.
Následující popis algoritmu je převzaný z \cite{HoneyBee}.

Včelí kolonie sestává z královny (nejlepší nalezené řešení), trubců (možná řešení), pracovníků (heuristická vylepšení)
a mladušek (potenciální řešení).
Na začátku je vytvořena populace trubců a nejlepší z nich je zvolený za královnu.
Poté se po daný počet opakování opakuje cyklus páření a vytváření nové generace.

Královně je přiřazena energie $e_{t0} = random \in <0.5, 1>$.
Poté královna náhodně navštivuje trubce, dokud se nespářila s předem daným počtem, nebo dokud jí nedojde energie ($e_t < \epsilon$).
Při každé návštěvě trubce je prav\-dě\-po\-dob\-nost spáření se s daným trubcem závislá na jejich fitness funkci $f$:
$p(queen, drone) = e^{\frac{f(queen) - f(drone)}{energy(t)}}$.
Pro tento problém je fitness funkcí počet porušení měkkých podmínek (královna tak má vždy nižší hodnotu než trubec).
Na začátku letu má královna vysokou energii, a pravděpodobnost páření je tak vyšší.
Po každém pokusu o páření je zmenšena energie královny ($e_{t+1} = e_{t} * \alpha$, $\alpha = 0.9$).

Po dokončení páření tvoří královna mladušky křížením své genetické informace s genetickou informací trubců, se kterými se královna spářila (jedna mladuška pro každého trubce).
Mladušce je přiřazena včela pracovník (heuristika) pro zlepšení fitness funkce mladušky.
Pokud je mladuška lepší než královna, vymění si s ní místo a stává se tedy novou královnou.

V obecné verzi algoritmu se pro další cyklus páření a vytváření mladušek trubci nemění, nicméně tento přístup vede k příliš rychlé konvergenci.
V modifikaci pro problém rozvrhnutí přednášek je předchozí generace trubců úplně nahrazena pozměněnými mladuškami (pomocí Kempe-chain struktur).
Tento modifikovaný přístup zabraňuje předčasné konvergenci algoritmu.

\subsection{Vytvoření počáteční populace}
Na začátku je potřeba získat několik řešení, která neporušují žádná tvrdá omezení.
Tato řešení jsou poté použita jako první generace trubců a nejlepší z nich také jako královna.

Pro vytvoření počáteční generace využíváme tří heuristik pro barvení grafu:
\begin{itemize}
  \item SD: přednášky jsou seřazeny vzestupně podle počtu vyučovacích hodin, do kterých je lze přiřadit
  \item LD: přednášky jsou seřazeny sestupně podle počtu všech přednášek, se kterými mají společné studenty
  \item LE: přednášky jsou seřazeny sestupně podle počtu studentů, kteří se této přednášky účastní
\end{itemize}
Všechny události nejdříve seřadíme podle těchto tří heuristik (s prioritami $SD > LD > LE$). Náhodně vybereme některou z platných vyučovacích hodin pro první přednášku v seznamu a zařadíme
ji do platné místnosti s nejmenší velikostí a následně nejmenším počtem vlastností.
Události znovu seřadíme a opakujeme náhodný výběr vyučovacích hodin a přiřazení do míst\-nos\-ti.
Pokud jsme nepřiřadili všechny události, a první událost v seznamu nemá žádnou přiřaditelnou vyučovací hodinu nebo ji nelze přiřadit do míst\-nos\-ti ve vybrané vyučovací hodině, dosavadní částečné řešení
zahodíme a začneme znovu.
Experimentálně se ukázalo, že není potřeba pro případ selhání zavádět prohledávání stavového prostoru (např. backtracking):
při generování po\-čá\-teč\-ních řešení se díky heuristikám jen vzácně vyskytnul případ, kdy nebylo možné zbývajícím hodinám přiřadit místo,
a nedochází tedy ně\-ko\-li\-ka\-ná\-sob\-né zahazování částečných řešení.

Tento přístup negarantuje, že existuje nějaké splnitelné řešení; v případě, že žádné neexistuje, zůstane algoritmus v nekonečné smyčce.
V žádné z testovaných úloh však tento problém nenastal.

\subsubsection{Křížení královny s trubcem}
Jako gen řešení je braná jedna vyučovací hodina, tj. všechny přednášky přiřazené do místností v jedné hodině v jednom dni.
Pro křížení královny s trubcem je použitý následující postup:
Mladuška je vytvořena jako kopie královny.
Z mladušky i trubce je náhodně vybráno 8 genů:
Pro každou dvojici genů $(q_k, t_k)$ se provede přesouvání přednášek z genu trubce do genu mladušky.
Po jednom se vybírají přiřazené přednášky z genu trubce a zařazují se do genu mladušky.
Pokud se již přednáška v genu mladušky vyskytuje, nebo by přemístěním této přednášky vznikl konflikt, nebo pokud nelze nalézt místnost, do které by šla tato přednáška zařadit,
předmět se nepřesouvá. Pokud nevznikne konflikt a lze nalézt místnost, je tato přednáška odstraněna ze svého původního umístění z genu mladušky.

\subsubsection{Zlepšení mladušky}
Po dokončení křížení je mladuška vylepšena pracovníkem.
Pro vylepšení podle použitého algoritmu z \cite{HoneyBee} je použité lokální prohledávání stavového prostoru.
Opakovaně se vybere náhodná přednáška a přesune se do náhodné vyučovací hodiny, která nenaruší tvrdá omezení.
Pokud tato změna zmenší počet porušení měkkých omezení, je přijata, pokud ne, je řešení zachováno ve své původní podobě.
Je provedený pevně daný počet iterací tohoto postupu.

\subsubsection{Kempe-chain stuktury}
Před použitím mladušek jako nových trubců je zvýšený jejich počet a jsou pozměněné pomocí Kempe-chain struktur.
Mladušky se duplikují, aby jejich celkový počet odpovídal požadované velikosti populace trubců a následně jsou promíchány.

Kempe-chain struktura (viz. \cite{Kempe})
pochází z barvení grafů: jde o největší souvislou komponentu sousedících uzlů, které jsou obarveny střídavě barvou $a$ a $b$. Poté je možné v této struktuře prohodit barvy $a$ a $b$.
Pro problém plánování přednášek lze najít Kempe-chain strukturu následovně:
Vybereme dvě vyučovací hodiny $x$ a $y$.
Náhodně vybereme některou z přednášek při\-řa\-ze\-nou do $x$ a vložíme ji do množiny $K_x$.\\
Definujeme predikát $Conflict(a, b) = \exists s \in S: (a \cup b) \subseteq c_s$, značící konflikt mezi přednáškami.
Dále definujeme predikát $Assigned(a, x)$, který platí, když je přednáška $a$ přiřazena do hodiny $x$.
Poté můžeme sestrojit obě množiny $K_x$ i $K_y$ následovně:
$$(v \in K_x \land Assigned(w, y) \land Conflict(v, w)) \implies w \in K_y$$
$$(v \in K_y \land Assigned(w, x) \land Conflict(v, w)) \implies w \in K_x$$
Poté můžeme z $x$ a $y$ uvolnit místnosti zabrané přednáškami z $K_x$ a $K_y$ a přiřadit do $x$ přednášky z $K_y$ a do $y$ přednášky z $K_x$.
Pokud nelze pro některou z přednášek nalézt volnou místnost, musí být celá změna zamítnuta a musí být případně zvolené jiné dvě vyučovací hodiny.

\section{Implementace}
Optimalizační nástroj byl implementovaný v \CC{}17. Pro změnu použitého překladače je možné v souboru \texttt{Makefile} změnit na prvním řádku obsah proměnné \texttt{CXX}.
Třídy modelující problém jsou definované a implementované v souboru \texttt{course\_timetabling.h}.
Samotný algoritmus, jak je popsaný v tomto článku, je implementovaný ve třídě \texttt{HbmoEtp} a jeho spuštění
implementuje metoda \texttt{run}.

Použité parametry algoritmu se nacházejí v souboru \texttt{config.h}. Úpravou tohoto souboru a novou kompilací
lze pozměnit parametry samotného algoritmu: počet iterací páření, počet trubců, počet mladušek (pro tuto implementaci by měl odpovídat čtvrtině počtu trubců),
počet křížených genů a počet iterací heuristického zlepšení mladušek. Nastavit lze také hranici kvality královny, po jejímž dosažení se algoritmus předčasně ukončí.
Počet trubců byl podle doporučení z \cite{HoneyBee} nastavený na 40, počet mladušek a křížených trubců na 10, počet křížených genů na 8, a počet iterací
vylepšování mladušek na 4000. Počet cyklů algoritmu byl pro většinu experimentů z časových důvodů snížený z doporučených 10000 na 2000.

Popsaný algoritmus je implementovaný tak, jak je popsaný v předchozí sekci, s následujícími modifikacemi:\\
Kvůli chybě při výpočtu pravděpodobnosti výběru trubce v původní implementaci bylo odhaleno, že pravděpodobnostní výběr trubce nemá pozorovatelný vliv na výslednou kvalitu řešení.
Byla proto možné z algoritmu odstranit proměnnou energie a jejího úbytku, a křížit královnu v každém cyklu s několika náhodně vybranými trubci (vždy úspěšně).\\
Pokud je nalezeno nové nejlepší řešení během fáze křížení, královna je zařazena do populace mladušek a
nejlepší mladuška nahrazuje královnu, a to i pro zbytek křížení v tomto cyklu.\\
Nová populace trubců je vytvořena duplikací mladušek a jejich modifikací pomocí Kempe-chain struktur.

Vstupně/výstupní formát problému a řešení byl převzaný z \url{http://sferics.idsia.ch/Files/ttcomp2002/oldindex.html}.
Program čte problém se standardního vstupu a výsledné nejlepší nalezené řešení vypisuje na standardní výstup.
V průběhu provádění algoritmu jsou na chybový výstup vypisované doprovodné informace o nejlepším dosaženém výsledku, především
počet konfliktů nejlepšího dosaženého řešení a na začátku které iterace jej bylo dosaženo.

Výsledky byly ověřené pomocí nástroje \texttt{tester}, který je kompilovaný ze zdrojového souboru dostupného
na adrese soutěže (soubor \texttt{checksln.cpp}), a program opravdu generuje kompletní a validní řešení.
Výpočet porušení měkkých omezení také souhlasí s informačními výstupy programu.

\section{Experimenty}
S výsledným optimalizačním nástrojem jsem prováděl dva typy experimentů. V prvním jsem provedl několik běhů programu pro společnou sadu vstupů a porovnal výsledky s ostatními projekty ze SNT.
V druhém jsem zvětšil počet iterací algoritmu na 10000 a sledoval jsem, jak se vyvíjí kvalita nalezeného řešení v závislosti na počtu iterací.

\subsection{Srovnání s ostatními algoritmy}
Experimenty byly prováděny na společné sadě vstupů (v souboru \texttt{tests.zip}). Pro každý vstupní soubor jsem provedl čtyři běhy, a nejlepší výsledek jsem vybral pro porovnání.
I s takto sníženým počtem iterací trvá jeden běh až půl hodiny (na procesoru Intel 7700HQ), pro
rychlé vyzkoušení doporučuji počet iterací kolem 100.

Porovnání mých výsledků (xvitra00) proběhlo proti následujícím algoritmům:
\begin{itemize}
  \item A Differential Evolution Algorithm (xvales02)
  \item Iterative Improvement Algorithm with Composite Neighbourhood Structures (xvales03)
  \item A hybrid evolutionary approach (xmarus06)
\end{itemize}
Výsledky porovnání jsou v tabulce \ref{tab:comp}.

\begin{table}[H]
\centering
\begin{tabular}{| c | c | c | c | c |}
\hline
vstupní soubor & \textbf{xvitra00} & \textbf{xvales02} & \textbf{xvales03} & \textbf{xmarus06} \\ \hline 
small1.tim & \textbf{11} & 47 & 31.2 & 33 \\ \hline
small2.tim & \textbf{21} & \textbf{21} & 34.5 & 32 \\ \hline
small3.tim & \textbf{16} & 38 & 28.0 & 26 \\ \hline
small4.tim & 45 & 53 & 41.7 & \textbf{42} \\ \hline
small5.tim & \textbf{20} & 37 & 32.6 & 30 \\ \hline
medium1.tim & \textbf{27} & 237 & 255.9 & 234 \\ \hline
medium2.tim & 154 & \textbf{62} & 126.2 & 132 \\ \hline
medium3.tim & \textbf{117} & 205 & 289 & 269 \\ \hline
medium4.tim & \textbf{26} & 66 & 173.1 & 170 \\ \hline
medium5.tim & \textbf{27} & 76 & 168.2 & 145 \\ \hline
large.tim & 369 & \textbf{342} & 408.4 & 468 \\ \hline
\end{tabular}
\caption{Porovnání počtu porušení měkkých omezení (nejlepší výsledek je zvýrazněn tučně)}
\label{tab:comp}
\end{table}

Pro většinu problémů z testovací sady přinesl algoritmus páření včel lepší nebo alespoň porovnatelné výsledky v porovnání s ostatní přístupy.
U osmi z jedenácti vstupů dosáhnul dokonce nejlepších výsledků ze všech porovnávaných přístupů (toto odpovídá výsledkům experimentů z článku \cite{HoneyBee}).
Pro vstup \texttt{medium2} ale dosáhnul horších výsledků než ostatní porovnávané algoritmy.

\subsection{Nejlepší řešení podle počtu iterací}
Pro tento experiment byl použitý vstupní soubor \texttt{large.tim} ze srovnávacích experimentů.
Algoritmus byl spuštěný s doporučeným množstvím 10000 iterací a byly sledované jeho průběžné výsledky z výstupů na chybový výstup.
Graf znázorňující body, kdy bylo dosažené nejlepší prozatímní řešení, je znázorněný na obrázku \ref{graph}.

\begin{figure}
\centering
\begin{tikzpicture}[scale = 1.3]
\begin{axis}[
    xlabel={Počet iterací},
    ylabel={Počet porušení měkkých omezení},
    xmin=0, xmax=10000,
    ymin=350, ymax=600,
    %xtick={0,20,40,60,80,100},
    %ytick={0,20,40,60,80,100,120},
    legend pos=north west,
    ymajorgrids=true,
    xmajorgrids=true,
    grid style=dashed,
    xmode=log,
    log ticks with fixed point
]
 
\addplot[only marks]
    coordinates {
    (0,783)
    (0,783)
    (1,592)
    (2,574)
    (3,564)
    (4,552)
    (5,544)
    (57,538)
    (58,533)
    (72,529)
    (73,520)
    (74,517)
    (129,514)
    (282,509)
    (462,502)
    (463,501)
    (521,499)
    (623,496)
    (643,494)
    (714,483)
    (715,478)
    (716,477)
    (745,469)
    (746,454)
    (747,450)
    (918,448)
    (919,447)
    (920,444)
    (921,440)
    (1035,437)
    (1036,433)
    (1077,432)
    (1093,431)
    (1094,430)
    (1165,422)
    (1166,417)
    (1279,416)
    (1457,413)
    (1458,409)
    (1459,406)
    (1799,399)
    (1820,397)
    (2224,396)
    (2555,394)
    (3038,391)
    (3835,389)
    (3967,388)
    (4084,386)
    (4229,379)
    (4826,376)
    (5976,373)
    (6108,371)
    (6648,366)
    (7094,364)
    (7236,362)
    (10000,362)
    };
 
\end{axis}
\end{tikzpicture}
\caption{Průběh nejlepšího řešení podle počtu iterací}
\label{graph}
\end{figure}

Z grafu můžeme pozorovat, že když se prozatímní řešení dostane z lo\-kál\-ní\-ho minima,
dosáhne algoritmus typicky během následujících dvou iterací dalšího zlepšení řešení.
Také můžeme pozorovat, že zlepšování řešení se zpomaluje, nedochází však k předčasné konvergenci (zlepšení se vyskytují i v pozdějších iteracích).

\section{Závěr}
%shrnout vysledky a experimenty
Byl implementovám algoritmus páření včel pro problém rozvrhnutí univerzitních přednášek.
Implementace byla zjednodušena odebráním prav\-dě\-po\-dob\-nos\-ti páření se královny s trubcem podle kvality trubce
bez pozorovaného dopadu na výsledky algoritmu.

Implementovaný algoritmus byl porovnaný proti některým ostatním optimalizačním algoritmům implementovaným v rámci předmětu SNT.
Výsledky porovnání odpovídaly zá\-vě\-rům původního článku, algoritmus překonával os\-tat\-ní přístupy pro většinu vstupů i přes zmenšení počtu iterací na jednu pětinu doporučené hodnoty.
Druhý experiment demonstruje, že algoritmus nekonverguje předčasně, a že i pro velké množství iterací nastává stálé zlep\-šo\-vá\-ní výsledku.

\makeatletter
    \def\@openbib@code{\addcontentsline{toc}{chapter}{Literatura}}

\begin{flushleft}
  \bibliography{doc}
\end{flushleft}
\end{document}