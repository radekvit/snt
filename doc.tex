\documentclass[12pt, a4paper]{article}
\usepackage[czech]{babel}
\usepackage[utf8]{inputenc}
\usepackage{amsmath}
\usepackage{amsthm}
\usepackage{amssymb}
\usepackage{enumitem}
\usepackage{algorithm}
\usepackage{algpseudocode}
\usepackage{mathtools}
\usepackage{listings}
\usepackage{color}
\PassOptionsToPackage{hyphens}{url}\usepackage{hyperref}

\usepackage{calrsfs}

\definecolor{light-gray}{gray}{0.85}
\lstset{
    numbers=left,
    breaklines=true,
    backgroundcolor=\color{light-gray},
    tabsize=2,
    basicstyle=\ttfamily,
}

\DeclareMathAlphabet{\pazocal}{OMS}{zplm}{m}{n}

\newcommand{\Lb}{\pazocal{L}}

%define Input and Output commands in algorithmicx
\algnewcommand\algorithmicinput{\textbf{Vstup:}}
\algnewcommand\Input{\item[\algorithmicinput]}

\algnewcommand\algorithmicoutput{\textbf{Výstup:}}
\algnewcommand\Output{\item[\algorithmicoutput]}

\let\oldemptyset\emptyset
\let\emptyset\varnothing
\let\oldepsilon\epsilon
\let\epsilon\varepsilon
\let\emptystring\varepsilon

\def\CC{{C\nolinebreak[4]\hspace{-.05em}\raisebox{.4ex}{\tiny\bf ++}}}
\def\CS{{\settoheight{\dimen0}{C}C\kern-.05em \resizebox{!}{\dimen0}{\raisebox{\depth}{\#}}}}

\author{Radek Vít\\ \texttt{xvitra00}}
\title{
	\includegraphics[scale=0.6]{FIT_barevne_PANTONE_CZ.pdf}\\
	SNT\\
	University Course Timetabling Problem:\\ Optimalizace pomocí algoritmu páření včel
}

\begin{document}
\newpage
\maketitle

\section{Úvod}
% proc tato prace
% muj prinos
% v jakem prostredi
\section{Popis problému}
Problém rozvrhnutí univerzitních přednášek je definovaný jako %TODO citace z HoneyBee
přiřazení daného počtu přednášek (také courses) do daného počtu vyučovacích hodin (také timeslots) a místností,
kde je splněná množina tvrdých omezení. %citovat Socha
V této práci se zaměřuji na tento konkrétní model úlohy, kterou tvoří:
\begin{itemize}
  \item množina přednášek $c_i$, $(i = 0, \dots, C)$
  \item množina vyučovacích hodin $t_n$, $(n = 0, \dots, 45)$ (9 pro každý den)
  \item množina místností $r_j$, $(j = 0, \dots, R)$
  \item množina vlastností místností $F$
  \item množina studentů $S$
\end{itemize}
Úlohou je přiřadit každou přednášku $c_i$ do hodiny $t_n$ a místnosti $r_j$ tak, aby byla splněna následující tvrdá omezení:
\begin{itemize}
  \item Každý student může mít v každé hodině nejvýše jednu přednášku.
  \item Přiřazená místnost musí splňovat vlastnosti vyžadované přednáškou.
  \item Přiřazená místnost musí mít dostatečnou kapacitu na všechny studenty účastnící se přednášky.
  \item V každé místnosti v jedné vyučovací hodině se může konat nejvýše jedna přednáška.
\end{itemize}
Kvalita řešení se poté hodnotí podle počtu porušení měkkých omezení:
\begin{itemize}
  \item Student by neměl mít přednášku v poslední hodině dne.
  \item Každý student by měl mít nejvýše 2 navazující přednášky.
  \item Student by měl mít za den více než jednu přednášku.
\end{itemize}
% ozdrojovana fakta
% pojmy ze SNT CITOVAT!!!
\section{Algoritmus páření včel pro problém rozvrhnutí univerzitních přednášek}
% zdrojuj
% TODO cituj honey-bee
Algoritmus páření včel je jedním z optimalizačních algoritmů inspirovaných přírodou.
Včelí kolonie sestává z královny (nejlepší nalezené řešení), trubců (možná řešení), pracovníků (heuristická vylepšení)
a mladušek (potenciální řešení).

Nejprve je vytvořena počáteční populace řešení. Nejlepší z počátečních řešení je zvoleno za královnu. Algoritmus sestává z opakovaného páření královny a vytváření mladušek.

\subsection{Vytvoření počáteční populace}
Na začátku je potřeba získat několik řešení, která neporušují žádná tvrdá omezení.
Tato řešení jsou poté použita jako první generace trubců a nejlepší z nich jako královna.

Pro vytvoření počáteční generace využíváme tří heuristik pro barvení grafu:
\begin{itemize}
  \item SD: přednášky jsou seřazeny vzestupně podle počtu vyučovacích hodin, do kterých je lze přiřadit
  \item LD: přednášky jsou seřazeny sestupně podle počtu všech přednášek, se kterými mají společné studenty
  \item LE: přednášky jsou seřazeny sestupně podle počtu studentů, kteří se této přednášky účastní
\end{itemize}
Všechny události nejdříve seřadíme podle těchto tří heuristik (s prioritami SD > LD > LE). Náhodně vybereme některou z platných vyučovacích hodin pro první přednášku v seznamu a zařadíme
ji do platné místnosti s nejmenší velikostí a následně nejmenším počtem vlastností.
Události znovu seřadíme a opakujeme náhodný výběr vyučovacích hodin a přiřazení do místnosti.
Pokud jsme nepřiřadili všechny události, a první událost v seznamu nemá žádnou přiřaditelnou vyučovací hodinu nebo ji nelze přiřadit do místnosti ve vybrané vyučovací hodině, dosavadní částečné řešení
zahodíme a začneme znovu. Experimentálně se ukázalo, že není potřeba zavádět prohledávání stavového prostoru, a že tento mechanismus je dostatečný. Při generování počátečních řešení se díky heuristikám jen vzácně vyskytlo několikanásobné zahazování částečných řešení.

Takto můžeme vytvořit každého z počátečních trubců.
\subsection{Páření královny}
Algoritmus páření včel simuluje přirozené chování včelí královny při páření.
Královna létá a páří se s trubci. Při úspěšném páření v sobě královna uchovává spermie trubců.
Královna končí páření, když jí dojde energie (v algoritmu se blíží nule); energie je po každém pokusu o páření násobena konstantou $\alpha = 0.9$.
V algoritmu se používá pravděpodobnostní model páření s trubci v závislosti jejich cenové funkci $f$: $p(queen, drone) = e^{\frac{f(queen) - f(drone)}{energy(t)}}$.
Cenová funkce $f$ je v případě problému rozvrhnutí počet porušení měkkých omezení. Na začátku letu má královna vysokou energii, a pravděpodobnost páření je tak vyšší.

\subsection{Vytvoření nové generace}
Po ukončení letů nastává vytváření mladušek. Tyto vznikají křížením nashromážděných spermií s genetickou informací královny.
Vzniklá mladuška je poté nakrmena (vylepšena) pracovníkem pro její zlepšení. Pokud je nejlepší mladuška lepší než současná královna, nahradí ji pro příští cyklus páření.
Zachování starých trubců vede k příliš rychlé konvergenci, kde se algoritmus zasekne na lokálním extrému.
V modifikaci pro tento problém se proto všichni trubci po páření zabijí a jsou nahrazeni modifikovanými mladuškami a případně starou královnou (pokud byla nahrazena).

\subsubsection{Křížení královny s trubcem}
Jako gen řešení je braná jedna vyučovací hodina, tzn. všechny přednášky přiřazené do místností v jedné hodině v jednom dni.
Pro křížení královny s trubcem je použitý následující postup:
Z královny a trubce je vybráno 8 genů a výsledek se vytvoří jako kopie královny.
Pro každou dvojici genů $(q_k, t_k)$ se provede přesouvání přednášek z genu trubce do nového genu královny.
Po jednom se vybírají přiřazené přednášky z genu trubce a zařazují se do genu královny.
Pokud se již přednáška v genu královny vyskytuje, nebo by přemístěním této přednášky vznikl konflikt, nebo pokud nelze nalézt místnost, do které by šla tato přednáška zařadit,
předmět se nepřesouvá. Pokud nevznikne konflikt a lze nalézt místnost, je tato přednáška odstraněna ze svého původního umístění z genu královny.

Po přesunutí všech vybraných genů vzniká mladuška.
\subsubsection{Zlepšení mladušky}
Po dokončení křížení je mladuška vylepšena pracovníkem.
V přístupu z %TODO
je použité lokální prohledávání stavového prostoru.
Opakovaně se vybere náhodná přednáška a přesune se do náhodné vyučovací hodiny, která nenaruší tvrdá omezení.
Pokud tato změna zmenší počet porušení měkkých omezení, je přijata, pokud ne, je řešení zachováno ve své původní podobě.
\subsubsection{Kempe-chain stuktury}
Před použitím mladušek jako nových trubců je zvýšený jejich počet a jsou pozměněné pomocí Kempe-chain struktur.
Mladušky se duplikují, aby jejich celkový počet odpovídal požadované velikosti populace trubců a následně jsou promíchány.

Kempe-chain struktura %todo cituj
pochází z barvení grafů: jde o největší souvislou komponentu sousedících uzlů, které jsou obarveny střídavě barvou $a$ a $b$. Poté je možné v této struktuře prohodit barvy $a$ a $b$.
Pro problém plánování přednášek lze najít Kempe-chain strukturu následovně:
Vybereme dvě vyučovací hodiny $x$ a $y$. Náhodně vybereme některou z přednášek přiřazenou do $x$ a vložíme ji do množiny $K_x$. Poté nalezneme všechny přednášky z $y$, které jsou v konfliktu s přednáškami z množiny $K_x$ a přidáme je do množiny $K_y$.
Poté střídavě přidáváme přednášky z $x$ do $K_x$, které mají konflikt s některou přednáškou z $K_y$, a přidáváme přednášky z $y$ do $K_y$, které mají konflikt s některou přednáškou z $K_x$.
Cyklus přerušíme, když se ustálí velikost $K_x$ i $K_y$. Poté můžeme z $x$ a $y$ uvolnit místnosti zabrané přednáškami z $K_x$ a $K_y$ a přiřadit do $x$ přednášky z $K_y$ a do $y$ přednášky z $K_x$.
Pokud nelze pro některou z přednášek nalézt volnou místnost, musí být celá změna zamítnuta a musí být případně zvolené jiné dvě vyučovací hodiny.

\section{Implementace}
\section{Experimenty}
\section{Závěr}
%shrnout vysledky a experimenty
\end{document}